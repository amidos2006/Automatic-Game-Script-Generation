This research presented a system to generate levels and rules for Puzzle Script engine. Also, it proposes several features to evaluate puzzle levels and games based on its solution sequence. Levels are generated regardless of the game rules. Two different techniques (Constructive and Genetic Approach) are tested for level generation. Based on the results, the constructive approach is used in rule generation. The rule generation is an extension to the work by Lim et al.\cite{puzzleScriptGeneration} without fixing the current level layout.\\\par

The constructive approach resulted in only 90\% playable levels which is enhanced using GA to reach 100\%. GA is tested using three different initialization methods (Random Initialization, Constructive Initialization, and Mixed Initialization). Random initialization produces different configuration than constructive algorithm but with lower playability equals to 75\%. The best two levels in GA approach are most of the time identical.\\\par

The levels are test against human players\footnote{http://www.amidos-games.com/puzzlescript-pcg/} and there is a correlation between the automated level score and human player score specially in BlockFaker and Sokoban. The high correlation in these games is the result of high performance of automated player in these games. This high correlation is a good indication that the proposed features measures the game playability and challenge.\\\par

Rule generation produced a group of rules that.....\\\par

As for future work, we aim to use our features to analyze the search space of puzzle level generators. Testing better optimization techniques instead of plan GA to increase level diversity. Testing our system with very huge level layouts. Improving the time of the quality of the automated player by trying. Testing Answer Set Programming as a way to speed up level generation time. Measuring level difficulty before generation. More tests to be made on rule generation.